\documentclass[]{article}

\usepackage[]{a4wide}

\setlength\parskip{\medskipamount}
\setlength\parindent{0pt}

\title{Measurement Equation Pipeline}
\author{Simon Perkins}

\begin{document}
\maketitle

\section{Introduction}

A simple formulation of the Radio Interferometry Measurement Equation (RIME) may be specified as follows:
$$
V_{pq} = G_p^H \left( \sum_k^N E_p^{(k)} K_p^{(k)} B^{(k)} K_q^{(k)} E_q^{(k)}  \right) G_q^H
$$
where
\begin{itemize}
\item $p$ and $q$ are the two antennae connecting a baseline $pq$.
\item $V_{pq}$ is the complex visibility pair generated by baseline antennae $p$ and $q$.
\item $G_p^H$ is the electronic gain matrix at antenna $p$.
\item $E_p^{(k)}$ is the source dependent effects matrix for antenna $p$ and source $k$.
\item $K_p^{(k)}$ is the phase delay matrix for antenna $p$ and source $k$.
\item $B^{(k)}$ is the brightness of source $k$.
\end{itemize}

The RIME equation relates describes how a model of the sky, consisting of $N$ point sources of varying brightness ($B^{(k)}$), and transformed by various instrumental ($G_p^H$, $K_p^{(k)}$) and environmental ($E_p^{(k)}$) effects, produces a set of simulated visibilities, $V_{pq}$ for baseline $pq$.

Other effects may be introduced through extra terms. This document will only consider those mentioned above for the sake of brevity.

Computing these terms is computationally expense: Recent work \cite{Baxter2012} shifts the computation from multi-core central processing units (CPUs) to General-purpose computing on graphics processing units (GPGPUs).

Raxter's approach shows good speedups but:

\begin{itemize}
\item everything in a single kernel?
\item separate kernels?
\item not a generalised framework?
\item add extra terms?
\end{itemize}

\section{Architecture}

\subsection{Pipeline}

\begin{itemize}
\item Each stage of pipeline invokes a kernel
\item Each pipe component transforms data somehow (frequency image). See \ref{sec:kernel_parallelism}.
\item 
\end{itemize}

\subsection{Kernel Parallelism}
\label{sec:kernel_parallelism}

antenna one $\times$ antenna two $\times$ frequency $\times$ time

\section{Implementation Details}

OpenCL vs CUDA?

\bibliography{../../bibliography/postdoc.bib}
\bibliographystyle{plain}

\end{document}